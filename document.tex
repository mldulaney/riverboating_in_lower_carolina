\documentclass[11pt, a5paper]{book}

% This is a transcription done by Mairi Dulaney of Seattle, WA

% chktex-file 18

\usepackage{caption}
\usepackage[english]{babel}
\usepackage[T1]{fontenc}
\usepackage[]{geometry}
\usepackage{textcomp}
\usepackage{gensymb}
\usepackage{graphicx}
\graphicspath{{images/}}
\usepackage[utf8]{inputenc}
\usepackage{titlesec}

\usepackage[activate={true,nocompatibility},final,
  tracking=true,kerning=true,factor=1100,stretch=10,shrink=10]{microtype}

\geometry{a5paper}
\nonfrenchspacing
\titleformat{\chapter}[display]
{\bfseries\Large}
{\centering
  \textsc{Chapter} \thechapter.} % label
{0.5ex}
{
  \vspace{1ex}
  \centering
  \scshape
}
[\vspace{-1ex}]

\titleformat{\section}[display]{\centering\bf}{\enspace}{0em}{}

\overfullrule=2cm


\begin{document}

\renewcommand{\thechapter}{\Roman{chapter}}
\title{Riverboating in Lower Carolina}
\author{F. Roy Johnson}
\date{1977}
\maketitle


\chapter{By Paddle, Ore, and Pole}

\section{Early Boats of Burden}

\textsc{Although} the dugout canoe, or log boat, has disappeared from
the rivers, creeks, and other waters of North Carolina's southeastern
counties, the watersheds of the Cape Fear, Black, and Northeast
rivers, it must be remembered as the common and more important vehicle
of travel and commerce of the early settlers of the region.  And to a
diminishing extent this versatile vehicle continued in common service
far into the nineteenth century.  As late as the 1950's a few
serviceable old moss cloaked log boats reposed in the South River
wilds of Bladen and Sampson counties and at remote landings of the
upper Waccamaw River in Columbus County.\par

The numerous waterways of the southeastern area --- the sounds, the
rivers, and the creeks --- were well suited for small boats.  These
avenues grew into the great channels of commerce and retained their
commanding position for nearly two centuries, until the railroads
became common and transportation needs moved westward.\par

Although much of the early travel was overland by horseback and by
foot on trails little better than Indian paths, virtually all the
commerce was by water.  However, by the Revolution, overland traffic
had increased; for the industrious settlers had built roads,
constructed causeways across swamps, bridged streams, and established
ferries.  This lessened littled the water freightage.  And up and down
the waterways there plied a great variety of craft loaded with
commodities.\par

Plantations, trading places, villages, and towns sprang up along the
streams; and the growth of the towns was determined to a large extent
by the volume of this commerce.  Chief of the towns were Fayetteville
at the head of navigation on the Cape Fear River, which funneled goods
down stream from the back country and received imported commodities,
and Wilmington, which exported and imported goods for the watersheds
of the Cape Fear River and her tributaries.\par

Not so fortunate was the town of Lisbon at the head of the Black River
in Sampson County.  Its merchants prospered and the town grew and
flourished on pole boat traffic.  But after steamboats came it went
neglected because of the river's shallows; and it dried up completely
when bypassed by the railroads.\par



\end{document}
